Este tipo de sistemas de ecuaciones también encuentra aplicación en el modelado de procesos estocásticos de difusión, donde una entidad se difunde típicamente
desde un lugar de mayor concentración hacia uno de menos. Es posible estudiar la evolución promedio de una densidad de partículas inicial resolviendo la ecuación de difusión de forma discreta
mediante Eliminación Gaussiana para sistemas tridiagonales \cite{langtangen-difusion}.

A partir de un vector inicial $u^{(0)}$ con magnitudes positivas, para cada punto discreto $i$, $u_{i}$ se atenuará y se dispersará hacia $u_{i-1}$ y $u_{i+1}$. En pos de mantener constante la suma total (es decir, la magnitud), la operación a utilizar deberá mantener iguales los ritmos de dispersión y atenuación. El operador laplaciano garantiza la conservación puesto que la suma de sus coeficientes es $1 - 2 + 1 = 0$. Considerando el incremento para el paso $k$ como una
fracción $\alpha$ del operador laplaciano de $u_{k}$, la ecuación resultante tiene la forma

\begin{equation} \label{eq:difusion}
	u_{i}^{(k)} - u_{i}^{(k-1)} = \alpha (u_{i-1}^{(k)} - 2 u_{i}^{(k)} + u_{i+1}^{(k)})
\end{equation}

En forma matricial, el sistema se puede expresar como $A u^{(k)} = u^{(k-1)}$. Resolviendo el sistema de forma iterativa, se obtiene la
evolución del vector u para $k = \{1, 2, 3, ..., m\}$. El resultado obtenido equivale a simular infinitas trayectorias individuales y analizar cómo se
distribuyen en promedio.