Los experimentos planteados muestran de manera clara cómo el algoritmo de Eliminación Gaussiana que aprovecha la estructura tridiagonal de las matrices utilizadas reduce notablemente los tiempos de ejecución en comparación a la versión tradicional, pasando de una complejidad computacional cúbica a una complejidad de orden lineal. Por otra parte, también demuestran que esta versión tridiagonal del algoritmo puede optimizarse mediante el precómputo de la diagonalización de la matriz, reduciendo aún más los tiempos de ejecución observados.

En cuanto al cálculo del Laplaciano discreto con los métodos propuestos, los resultados observados fueron los esperados, mostrando nuestra implementación un comportamiento idéntico al presentado por la cátedra.

Por último, se utilizaron los algoritmos desarrollados en el presente trabajo para experimentar la simulación de un proceso de difusión. Se demostró el rol que juega el parámetro $\alpha$ en la velocidad de la difusión de los valores inicialmente concentrados en el centro del vector.