Un caso de aplicación de sistemas tridiagonales a procesamiento de grafos e imágenes es el cálculo del operador de Laplace o Laplaciano en su versión discreta. Para el caso unidimensional la operación está dada por:

\begin{equation} \label{eq:laplaciano}
	u_{i-1} - 2u_{i} + u_{i+1} = d_{i}
\end{equation}

siendo $u_{-1} = u_{n} = 0$. El sistema en forma matricial tiene la forma:

\[
\begin{bmatrix}
-2 & 1 &  &  & 0 \\
1 & -2 & 1 &  &  \\
 & 1 & -2 & \ddots &  \\
 &  & \ddots & \ddots & 1 \\
0 &  &  & 1 & -2 \\
\end{bmatrix} 
\begin{bmatrix}
u_{1} \\
u_{2} \\
u_{3} \\
\vdots \\
u_{n} \\
\end{bmatrix}
=
\begin{bmatrix}
d_{1} \\
d_{2} \\
d_{3} \\
\vdots \\
d_{n} \\
\end{bmatrix}
\]

Aprovechando el sistema matricial tridiagonal asociado al problema, puede aprovecharse la adaptación de Eliminación Gaussiana para sistemas tridiagonales para hallar el vector $u$ asociado a cada vector $d$.