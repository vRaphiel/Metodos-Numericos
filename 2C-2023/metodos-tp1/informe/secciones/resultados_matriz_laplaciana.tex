La segunda parte del análisis consiste en interpretar los resultados de la ejecución del cálculo de la matriz Laplaciana. Se sabe que, aplicado a un vector, este opera como la derivada segunda del mismo.
Es por tal motivo que resulta de interés ver si, mediante un sistema de ecuaciones lineales, se puede encontrar la función que represente a la ecuación $ (d^{2}/dx^{2})u = d $.
Por tal motivo, se definirán distintos $d$ que nos permitan encontrar la función
$$  d_{i} = \begin{cases}
      0 \\
      4/n \ \textit{si} \ i =  \lfloor n/2 \rfloor + 1
   \end{cases}
$$
$$ d_{i} = 4/n^{2} $$
$$ d_{i} = (-1 + 2i/(n - 1))12/n^{2}$$

Se utilizará un valor de $n = 101$ y se procederá a realizar un gráfico de los resultados con el objetivo de ver las funciones que pueden representar.
El resultado se observa en la figura \ref{fig:difusion}.

\begin{figure}[H]
   \centering
   \includesvg[scale=0.5]{graficos/laplaciana.svg}
   \caption{Derivada segunda de los u para distintos d}
   \label{fig:difusion}
\end{figure}
