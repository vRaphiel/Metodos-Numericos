Un sistema de $n$ ecuaciones con $n$ incógnitas puede escribirse de forma matricial como $Ax\ =\ b$, siendo $A \in \mathbb{R}^{n \times n}$ la matriz cuadrada con los coeficientes de las incógnitas, $x \in \mathbb{R}^{n \times 1}$ el vector de incógnitas y $b \in \mathbb{R}^{n \times 1}$ el vector con los coeficientes independientes. Concatenando el vector $b$ a derecha de la matriz $A$ obtenemos la matriz aumentada $A^{\prime} \in \mathbb{R}^{n \times n+1}$.

Los sistemas de ecuaciones con una única solución pueden ser resueltos mediante el algoritmo de Eliminación Gaussiana \cite{burden-gauss} con una complejidad computacional de orden $\mathcal{O}(n^3)$. Este toma como input la matriz aumentada $A^{\prime}$ y en cada paso procede a anular los elementos por debajo de la diagonal de cada columna mediante operaciones de fila, siempre que sea posible, obteniendo como resultado una matriz escalonada. El elemento ubicado en la diagonal de la matriz que se utiliza en cada paso como base de las operaciones se llama pivot. Por último, se despejan y reemplazan hacia atrás cada una de las incógnitas.

Uno de los casos en que el algoritmo no es capaz de devolver una solución es cuando, en alguno de los pasos, se halla un elemento nulo en la diagonal. En estos casos puede aplicarse pivoteo parcial o total, consistente en permutar filas y/o columnas para desbloquear el paso y continuar con el procedimiento. El pivoteo permite también reducir el error numérico de la resolución, al optar en cada paso por el mayor elemento de la matriz a modo de pivot.